\documentclass[11pt]{article}

% use packages
\usepackage[utf8]{inputenc}
\usepackage{amsmath}
\usepackage{amsthm}
\usepackage{amsfonts}
%\usepackage{amscd}
\usepackage{amssymb}
\usepackage{graphicx}
\usepackage{mathtools}
\usepackage{natbib}
\usepackage{enumitem}
\usepackage{url}
\usepackage{authblk}
\usepackage{bm}
\usepackage[usenames]{color}
\usepackage{hyperref}
\usepackage{geometry}
\usepackage{caption}
\usepackage{float}
\usepackage{tikz}
\usepackage[linesnumbered, ruled,vlined]{algorithm2e}
% margin setup
\geometry{margin=1in}


% function definition
\newcommand{\R}{\mathbb{R}}
\newcommand{\w}{\textbf{w}}
\newcommand{\x}{\textbf{x}}
\newcommand{\X}{\textbf{X}}
\newcommand{\Y}{\textbf{Y}}
\newcommand{\Hist}{\mathcal{H}}
\def\mbf#1{\mathbf{#1}} % bold but not italic
\def\ind#1{\mathrm{1}(#1)} % indicator function
\newcommand{\simiid}{\stackrel{iid}{\sim}} % IID 
\def\where{\text{ where }} % where
\newcommand{\indep}{\perp \!\!\! \perp } % independent symbols
\def\cov#1#2{\mathrm{Cov}(#1, #2)} % covariance 
\def\mrm#1{\mathrm{#1}} % remove math
\newcommand{\reals}{\mathbb{R}} % Real number symbol
\def\t#1{\tilde{#1}} % tilde
\def\normal#1#2{\mathcal{N}(#1,#2)} % normal
\def\mbi#1{\boldsymbol{#1}} % Bold and italic (math bold italic)
\def\v#1{\mbi{#1}} % Vector notation
\def\mc#1{\mathcal{#1}} % mathical
\DeclareMathOperator*{\argmax}{arg\,max} % arg max
\DeclareMathOperator*{\argmin}{arg\,min} % arg min
\def\E#1{\mathrm{E}(#1)} % Expectation symbol
\def\var#1{\mathrm{Var}(#1)} % Variance symbol
\def\checkmark{\tikz\fill[scale=0.4](0,.35) -- (.25,0) -- (1,.7) -- (.25,.15) -- cycle;} % checkmark
\newcommand\red[1]{{\color{red}#1}}


\newcommand{\norm}[1]{\left\lVert#1\right\rVert} % A norm with 1 argument
\DeclareMathOperator{\Var}{Var} % Variance symbol

\newtheorem{cor}{Corollary}
\newtheorem{lem}{Lemma}
\newtheorem{thm}{Theorem}
\newtheorem{defn}{Definition}
\newtheorem{prop}{Proposition}
\theoremstyle{definition}
\newtheorem{remark}{Remark}
\hypersetup{
  linkcolor  = blue,
  citecolor  = blue,
  urlcolor   = blue,
  colorlinks = true,
} % color setup

% proof to proposition 
\newenvironment{proof-of-proposition}[1][{}]{\noindent{\bf
    Proof of Proposition {#1}}
  \hspace*{.5em}}{\qed\bigskip\\}
% general proof of corollary
  \newenvironment{proof-of-corollary}[1][{}]{\noindent{\bf
    Proof of Corollary {#1}}
  \hspace*{.5em}}{\qed\bigskip\\}
% general proof of lemma
  \newenvironment{proof-of-lemma}[1][{}]{\noindent{\bf
    Proof of Lemma {#1}}
  \hspace*{.5em}}{\qed\bigskip\\}

\allowdisplaybreaks



% title
\title{Supplemental Materials for ``Minimizing post shock forecasting error using disparate information''}
\author{Jilei Lin\thanks{jileil2@ilinois.edu}}
\author{Ziyu Liu\thanks{ziyuliu3@illinois.edu}}
\author{Daniel J. Eck\thanks{dje13@illinois.edu}}
\affil{Department of Statistics, University of Illinois at Urbana-Champaign}

%%% New version of \caption puts things in smaller type, single-spaced 
%%% and indents them to set them off more from the text.
\makeatletter
\long\def\@makecaption#1#2{
  \vskip 0.8ex
  \setbox\@tempboxa\hbox{\small {\bf #1:} #2}
  \parindent 1.5em  %% How can we use the global value of this???
  \dimen0=\hsize
  \advance\dimen0 by -3em
  \ifdim \wd\@tempboxa >\dimen0
  \hbox to \hsize{
    \parindent 0em
    \hfil 
    \parbox{\dimen0}{\def\baselinestretch{0.96}\small
      {\bf #1.} #2
      %%\unhbox\@tempboxa
    } 
    \hfil}
  \else \hbox to \hsize{\hfil \box\@tempboxa \hfil}
  \fi
}
\makeatother



\begin{document}

\maketitle




\vspace{.2cm}
\begin{algorithm}[H]
\SetAlgoLined \label{boots}

\SetKwInOut{Input}{Input}
\Input{$B$ --  the number of parametric bootstraps  \\  $\{(y_{i,t}, \mbf{x}_{i,t})\colon i = 2,\ldots, n+1, t = 0, \ldots, T_i\}$ -- the data\\ $\{T_i^* \colon i = 1, \ldots, n+1\}$ -- the time point just before the shock\\$\{\hat{\varepsilon}_{i, t} \colon t = 1, \ldots, T_i\}$ -- the collection of  residuals for $t = 1,\ldots, T_i$ \\ $\{\hat{\eta}_i, \hat{\alpha}_i, \hat{\phi}_i, \hat{\theta}_i, \hat{\beta}_i\colon i = 2, \ldots, n+1\}$ -- the OLS estimates \\
}

\KwResult{The sample mean, and sample variance of bootstrapped adjustment estimator, inverse-variance weighted estimator, and weighted-adjustment estimator.}

  \For{$b = 1:B$}{
  \For{$i = 2, \ldots, n+1$}{
  
    Sample with replacement from $\{\hat{\varepsilon}_{i, t} \colon t = 1, \ldots, T_i\}$ to obtain $\{\hat{\varepsilon}_{i, t}^{(b)} \colon t = 1, \ldots, T_i\}$
    
     Define $y_{i,0}^{(b)}=y_{i,0}$
    
    \For{$t = 1,\ldots, T_i$}{
     

         
     Compute $y_{i,t}^{(b)}=\hat{\eta}_i +\hat{\alpha}_i 1(t=T_i^*+1)+\hat{\phi}_iy_{i,t-1}^{(b)}+\theta_i'\mbf{x}_{i,t} + \beta_i'\mbf{x}_{i, t-1}+\hat{\varepsilon}_{i,t}^{(b)}$
    }
    Compute $\hat{\alpha}_i^{(b)}$ based on OLS estimation of the parameters in $\mc{M}_1$ (\textbf{does this work for the other models?}) with $\{(y_{i,t}^{(b)}, \mbf{x}_{i,t}) \colon t= 0, \ldots, T_i\}$
  }
   Compute the $b$th shock-effect estimate $\hat{\alpha}^{(b)}_{\rm est}$ for $\mrm{est} \in \{\mrm{adj}, \mrm{wadj}, \mrm{IVW}\}$
  }
   Compute the sample mean, and sample variance of $\{\hat{\alpha}_{\rm est}^{(b)} \colon b = 1, \ldots, B\}$ for $\mrm{est} \in \{\mrm{adj}, \mrm{wadj}, \mrm{IVW}\}$
 \caption{Parametric bootstrap for approximation for  mean and  variance of shock-effect estimators of $\alpha_1$.}
\end{algorithm}
\vspace{.2cm}




\bibliographystyle{plainnat}
\bibliography{synthetic-prediction-notes}

\end{document}



