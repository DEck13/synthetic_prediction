% Template for articles submitted to the International Journal of Forecasting
% Further instructions are available at www.ctan.org/pkg/elsarticle
% You only need to submit the pdf file, not the source files.
% If your article is accepted for publication, you will be asked for the source files.

\documentclass[11pt,3p,review,authoryear]{elsarticle}


\journal{International Journal of Forecasting}
\bibliographystyle{model5-names}
\biboptions{longnamesfirst}
% Please use \citet and \citep for citations.


\begin{document}

\begin{frontmatter}

\title{Minimizing post-shock forecasting error through aggregation of outside information}

  \author[uiuc]{Jilei Lin\corref{cor}}
   \ead{jileil2@illinois.edu}

 \author[uiuc]{Daniel J Eck}
 \address[uiuc]{Department of Statistics, University of Illinois at Urbana-Champaign\\
 725 S. Wright St., Champaign, IL, 61820, U.S.}
 \ead{dje13@illinois.edu}
  \cortext[cor]{Corresponding author}


\begin{abstract}
 We develop a forecasting methodology for providing credible forecasts for time series that have recently undergone a shock. We achieve this by borrowing knowledge from other time series that have undergone similar shocks for which post-shock outcomes are observed. Three shock effect estimators are motivated with the aim of minimizing average forecast risk. We propose risk-reduction propositions that provide conditions that establish when our methodology works. Bootstrap and leave-one-out cross validation procedures are provided to prospectively assess the performance of our methodology. Several simulated data examples, and a real data example of forecasting Conoco Phillips stock price are provided for verification and illustration.
\end{abstract}

\begin{keyword}
  Data Integration \sep 
 Prospective forecasting \sep 
 Risk reduction \sep 
Residual bootstrap \sep 
Cross validation 
\end{keyword}

\end{frontmatter}




\end{document}