\documentclass[11pt]{article}


\usepackage{tikz}
\usepackage{geometry}
\usepackage{graphicx}
\usepackage{color}
\usepackage[export]{adjustbox}


\setlength{\parindent}{0em}
\setlength{\parskip}{1em}


\usepackage{geometry}
\geometry{margin=1in}

\renewcommand*\arraystretch{1.05}

\begin{document}

\footnotesize
%\color{lightblue}
\begin{tikzpicture}
  \node at (0,0) {\includegraphics[height=1.1in, keepaspectratio = true]{UI.png}};
  \node at (13.4, -0.05) {\vspace{1cm}\begin{tabular}{l}
      \\ \sc Jilei Lin \\
       \it Graduate Student \\
       \it Department of Statistics \\
       University of Illinois \\
       Daniels Hall, Room 0506A \\
	   1010 W. Green St \\
	   Urbana, IL 61820 \\
       \texttt{jileil2@illinois.edu} \\ 
       \end{tabular} };
\end{tikzpicture}

\normalsize 


Dear editor,

We are excited to submit our manuscript ``Minimizing post-shock forecasting error through aggregation of outside information'' for possible publication at the  International Journal of Forecasting. 

In this manuscript, we  develop a forecasting methodology for providing credible forecasts for time series that are known to  experience a shock in the future. It is a general methodology that has various potential real-life applications in forecasting during  \emph{unprecedented} time. We achieve this by borrowing knowledge from other time series that have undergone similar shocks for which post-shock outcomes are observed. We construct three shock effect estimators  with the aim of minimizing average forecast risk. We propose risk-reduction propositions that provide conditions that establish when our methodology works. Bootstrap and leave-one-out cross validation procedures are provided to prospectively evaluate those conditions and estimate corresponding correctness. We use a real data example of forecasting Conoco Phillips stock price during the time  of   oil price control and COVID-19 pandemicto verify  our methodology.

We confirm that this manuscript is only under consideration at the International Journal of Forecasting. Thank you for your consideration.

Sincerely,

Jilei Lin \\
Daniel J. Eck 




\end{document}