\documentclass[11pt]{article}
\usepackage[utf8]{inputenc}
\usepackage{amsmath}

\usepackage{geometry}
\geometry{margin=1in}

\newcommand{\w}{\textbf{w}}
\newcommand{\x}{\textbf{x}}
\newcommand{\X}{\textbf{X}}
\newcommand{\Hist}{\mathcal{H}}

\newcommand{\norm}[1]{\left\lVert#1\right\rVert}
\newcommand{\indep}{\perp\!\!\!\perp}

\DeclareMathOperator{\E}{E}
\DeclareMathOperator{\Var}{Var}

\title{Synthetic prediction methods for minimizing post shock forecasting error}
\author{Daniel J. Eck and Soheil Eshghi}
\date{May 2019}

\begin{document}


\maketitle
\begin{abstract}
    We seek to develop a forecasting methodology for time series data that is 
    thought to have undergone a shock which has origins that have not been 
    previously observed.  We still can provide credible forecasts for a time 
    series in the presence of such systematic shocks by drawing from disparate 
    time series that have undergone similar shocks for which post-shock 
    outcome data is recorded.  These disparate time series are assumed to have 
    mechanistic similarities to the time series under study but are otherwise 
    independent (Granger noncausal).  The inferential goal of our forecasting 
    methodology is to supplement observed time series data with post-shock 
    data from the disparate time series in order to minimize average forecast 
    risk. 
\end{abstract}


\section{Setting}
Suppose that an analyst is interested in forecasting a real-valued time series 
$y_1, y_2, \ldots$.  Given each time point $t \geq 1$, let $\x_t$ be the 
(possibly multivariate) information variable vector revealed prior to the 
observation of $y_t$.  To gauge the performance of a procedure that produces forecasts 
$\{\hat y_t, t= 1,2,\ldots\}$ given time horizon $T$, we consider the average 
forecast risk
$$
  R_T = \frac{1}{T}\sum_{t=1}^T\E(y_t - \hat y_t)^2
$$
in our analyses.


\section{Dynamic panel model}
\label{sec:dpm}
We consider a dynamic panel data model similar with autoregressive structure.  
For simplicity, we will consider the following simplified version with one covariate:
\begin{equation} \label{DPM}
  y_{i,t} = \eta_i + \alpha_iD_{i,t} + \phi_i y_{i,t-1} + \theta_i \x_{i,t} 
    + \varepsilon_{i,t}, 
\end{equation}
where $i = 1,\ldots,J+1$ and $t = 1,\ldots,T$ and $D_{i,t} = 1(t > T_i^*)$, 
$T_i^* < T$.  For simplicity we will assume that $T_i^* = T$ for all $i = 1$,
$\ldots$, $J$.  We will consider a simple random effects structure where 
\begin{align*}
  \eta_i &\overset{iid}{\sim} N(0, \sigma_\eta^2), 
    \qquad i = 1,\ldots J+1, \\
  \alpha_i &\overset{iid}{\sim} N(\mu_\alpha, \sigma_\alpha^2), 
    \qquad i = 1,\ldots J+1, \\
  \phi_i &\overset{iid}{\sim} U(-1,1), 
    \qquad i = 1,\ldots J+1, \\
  \theta_i &\overset{iid}{\sim} U(-1,1), 
    \qquad i = 1,\ldots J+1, \\
  \varepsilon_{i,t} &\overset{iid}{\sim} N(0, \sigma^2), 
    \qquad i = 1,\ldots J+1;\; t = 1, \ldots T, \\
  \eta &\indep \alpha \indep \phi \indep \theta \indep \varepsilon;
\end{align*}
Without loss of generality, we will suppose that we want to forecast for 
$y_{1, T^* + 1}$ where we have no observations for $y_{1,t}, t > T^*$. 
Inference about the random effects parameters is not of interest in this  
forecasting context.  Conditional on all regression parameters 
(except $\alpha$?), previous responses, and covariates, model \eqref{DPM} 
has distribution
$$
  y_{i,t} 
    \sim N(\eta_i + \alpha_iD_{i,t} + \phi_i y_{i,t-1} + \theta_i\x_{i,t}, \sigma^2).
$$
All parameters will be estimated with OLS. In particular, let 
$\hat{\alpha}_i$, $i = 2,\ldots,J+1$ be the OLS estimate of $\alpha_i$ 
and define the adjusted $\alpha$ plugin estimator for time series $i=1$ by,
\begin{equation} \label{adjusted}
  \hat{\alpha}_{\text{adj}} = \frac{1}{J}\sum_{i=2}^{J+1}\hat{\alpha}_i
\end{equation}
where the $\hat{\alpha}_i$s are MLEs of all of the $\alpha_i$s.  
We can use $\hat{\alpha}_{\text{adj}}$ as an estimator for the unknown 
$\alpha_1$ term for which no meaningful estimation information otherwise 
exists. 


Consider the candidate forecasts: 
\begin{align*}
  &\text{Forecast 1}: \hat y_{1,T^*+1}^1 = \hat\eta_1 
    + \hat\phi_1 y_{1,T^*} + \hat\theta_1 \x_{1,T^*+1}, \\
  &\text{Forecast 2}: \hat y_{1,T^*+1}^2 = \hat\eta_1 
    + \hat\phi_1 y_{1,T^*} + \hat\theta_1 \x_{1,T^*+1} + \hat{\alpha}_{\text{adj}},
\end{align*}
where $\hat\eta_1$, $\hat\phi_1$, and $\hat\theta_1$ are MLEs of 
$\eta_1$, $\phi_1$, and $\theta_1$ respectively.  
The two forecasts do not differ in their predictions for 
$y_{t}$, $t = 1,\ldots T^*$.  They only differ in predicting $y_{T^*+1}$.  \\


{\bf We want to determine when either $\hat y_{1,T^*+1}^1$ or 
$\hat y_{1,T^*+1}^2$ minimizes the average forecast risk.}


The two forecasts do not differ in their predictions for 
$y_{t}$, $t = 1,\ldots T^*$.  They only differ in predicting 
$y_{T^*+1}$.  The forecast risk for $y_{T+1,2}$ is: 
\begin{align*}
  %&R_{T^*+1, 2} - R_{T^*, 2} = 
  &\E(\hat y_{1,T^*+1}^2 - y_{1,T^*+1}^2)^2 \\
  &\qquad= \E\left\{ \hat\eta_1 + \hat\phi_1 y_{1,T^*} + \hat\theta_1 \x_{1,T^*+1} 
    + \hat{\alpha}_{\text{adj}}
      - (\eta_1 + \phi_1 y_{1,T^*} + \theta_1 \x_{1,T^*+1} + \alpha_1 + \varepsilon_{1,T^*+1}) \right\}^2 \\
  &\qquad= \E\left\{ (\hat\eta_1 + \hat\phi_1 y_{1,T^*} + \hat\theta_1 \x_{1,T^*+1}     
      -\eta_1 - \phi_1 y_{1,T^*} - \theta_1 \x_{1,T^*+1} - \alpha_1 - \varepsilon_{1,T^*+1}) 
        + \hat{\alpha}_{\text{adj}} \right\}^2 \\      
  &\qquad= R_{T, 1} + \E(\hat{\alpha}_{\text{adj}}^2) - 2\E\left\{(\hat\eta_1 + \hat\phi_1 y_{1,T^*} 
    + \hat\theta_1 \x_{1,T^*+1} -\eta_1 - \phi_1 y_{1,T^*} - \theta_1 \x_{1,T^*+1} - \alpha_1 - \varepsilon_{1,T^*+1})
      \hat{\alpha}_{\text{adj}} \right\} \\
  &\qquad= R_{T, 1} + \E(\hat{\alpha}_{\text{adj}}^2) - 2\E(\hat{\alpha}_{\text{adj}})\E(\alpha_1) 
    - 2\E\left\{(\hat\eta_1 + \hat\phi_1 y_{1,T^*} + \hat\theta_1 \x_{1,T^*+1} -\eta_1 - \phi_1 y_{1,T^*} 
      - \theta_1 \x_{1,T^*+1})\hat{\alpha}_{\text{adj}} \right\} \\
  &\qquad= R_{T, 1} + \Var(\hat{\alpha}_{\text{adj}}) - \mu_{\alpha}^2 
    - 2\E\left(\hat\eta_1 + \hat\phi_1 y_{1,T^*} + \hat\theta_1 \x_{1,T^*+1} -\eta_1 - \phi_1 y_{1,T^*} 
      - \theta_1 \x_{1,T^*+1}\right)\mu_{\alpha},
\end{align*}
where 
$$
  \E(\hat{\alpha}_{\text{adj}}) = \E\left\{\E(\hat{\alpha}_{\text{adj}}| \Hist )\right\}
    = \E\left(\frac{1}{J}\sum_{i=2}^{J+1}\alpha_i\right) 
    = \mu_{\alpha},
$$
and 
$$
  \Hist = \{(\eta_i, \phi_i, \theta_i, \alpha_i, \x_{i,t}, y_{i,t}); 
    i = 2,\ldots,J+1, t = 1,\ldots,T\}.
$$
Define
$$
  \Hist_1 = \{\eta_1, \phi_1, \theta_1, \alpha_1, \x_{1,T^*+1}, (\x_{1,t}, y_{1,t});  
    t = 1,\ldots,T^*\}.
$$
Notice that 
\begin{align*}
  &\E\left(\hat\eta_1 + \hat\phi_1 y_{1,T^*} + \hat\theta_1 \x_{1,T^*+1} -\eta_1 - \phi_1 y_{1,T^*} 
    - \theta_1 \x_{1,T^*+1}\right) \\
  &\qquad= \E\left\{\E\left(\hat\eta_1 + \hat\phi_1 y_{1,T^*} + \hat\theta_1 \x_{1,T^*+1} -\eta_1 - \phi_1 y_{1,T^*} 
    - \theta_1 \x_{1,T^*+1}\right)|\Hist\cup\Hist_1\right\} = 0,
\end{align*}
(\textbf{is the above true?}) and that 
\begin{align*}
  \Var(\hat{\alpha}_{\text{adj}}) &= \E\left\{\Var(\hat{\alpha}_{\text{adj}}|\Hist)\right\} 
    + \Var\left\{\E(\hat{\alpha}_{\text{adj}}|\Hist)\right\} \\
  &= \E\left\{\Var(\hat{\alpha}_{\text{adj}}|\Hist)\right\} + \frac{\sigma_{\alpha}^2}{J} \\
  &= \frac{1}{J^2} \sum_{i=2}^{J+1} \E\left\{\Var(\hat{\alpha}_j|\Hist)\right\} + \frac{\sigma_{\alpha}^2}{J} \\
  &= \frac{1}{J^2(T-1)}\sum_{i=2}^{J+1} \E\left\{\frac{s_{e,j}^2}{(1-R^2_j)s_{\alpha_j}^2}\right\} 
    + \frac{\sigma_{\alpha}^2}{J},
\end{align*}
where $s_{e,j}^2$ is the estimated variance of model $j$, $R^2_{\alpha_j}$ is 
the multiple $R^2$ obtained from regressing the shock indicator $D$ on the 
other regressors, and $s_j^2$ is... .

Forecast 2 has lower forecast risk than Forecast 1 when,  
\begin{equation} \label{risk}
  \Var(\hat{\alpha}_{\text{adj}}) - \mu_{\alpha}^2 
    - 2\E\left(\hat\eta_1 + \hat\phi_1 y_{1,T^*} + \hat\theta_1 \x_{1,T^*+1} -\eta_1 - \phi_1 y_{1,T^*} 
      - \theta_1 \x_{1,T^*+1}\right)\mu_{\alpha} < 0.
\end{equation}
Putting everything together, we see that \eqref{risk} is equivalent to
$$
  \frac{1}{J^2(T-1)}\sum_{i=2}^{J+1} \E\left\{\frac{s_{e,j}^2}{(1-R^2_j)s_{\alpha_j}^2}\right\} 
    + \frac{\sigma_{\alpha}^2}{J} < \mu_{\alpha}^2.
$$
Forecast 2 is preferable to Forecast 1 asymptotically in both $T$ and $J$ 
whenever $\mu_{\alpha} \neq 0$. In finite samples, Forecast 2 is preferable to 
Forecast 1 when the $\mu_{\alpha}$ is large relative to its variability and 
overall regression variability.  














\section{Synthetic prediction via by synthetic control method}


{\bf Taken from Soheil's notes (begin).}
Assume we have covariate-observation pairs $(y_i, \x_i)$ for $n$ points, and 
that $\x_1$, the covariates of the intervened unit, lies within the convex hull 
of $(\x_i)_{i=1}^n$. One way to match covariates is to pick a vector $\w$ such 
that $\w'1=1$ and $\norm{\x_1 - \X\w}$, where $\X$ is the matrix of available 
covariates (arranged in column form), is minimized.

One can incorporate a notion of locality by penalizing the use of covariates 
far from the covariates of interest, through incorporating the following term 
in either a constraint or as part of the objective:
$\sum_{i=1}^n \w_i \norm{\x_1 - \x_i}$.

The use of the transformed linear model (linking covariates to observations of 
outcomes) to create the counterfactual show our belief in the model 
specification. Under these conditions, the prediction of the model for the 
outcome at covariates $\x_i$ based on all other observed covariates $\x_{-i}$ 
(if it is within the convex hull) should closely match its observed outcome.  
Any discrepancy should make us less confident in the observation of the 
outcome (if we hold the model specification to be correct), so we would want 
$y_i$ to play less of a role in the creation of the counterfactual in that 
scenario.  Thus, we will put an upper limit on $\w_i$, the weight assigned to 
the covariate-observation pair by a function of the discrepancy between $y_1$ 
and the created synthetic prediction for it $\hat{y}_i$ from $\x_{-i}$.  
{\bf Taken from Soheil's notes (end).} \\

%The question of how to minimize variance in the $Yw$ can be addressed in 
%multiple ways. One is to incorporate the weighted sample variance 
%(weighted by $w$) into a constraint or as part of the objective.  However, 
%this is not the right way to address this issue, as it abstracts out the 
%dependence on the covariates. Thus, it is likely that we will need to create a 
%covariate-weighted variance measure for the synthetic controls.
%{\bf Taken from Soheil's notes (end).}

Prediction via this synthetic control method can be applied to the dynamic 
panel models in Section~\ref{sec:dpm}.  In this case, the individual level 
shock effects $\alpha_i$ are correlated with the covariate information 
$\x_i$.  We consider the model 
\begin{equation} \label{shockmodel}
  \alpha_{i} = \mu_{\alpha_i} + \beta\x_i + \gamma_{i}
\end{equation}
for the shock effects, where $\gamma_{i}$ are iid with 
$\E(\gamma_i) = 0$ and $\Var(\gamma_i) = \sigma^2_\gamma$ and are 
independent of all other random effects.  
The specification in \eqref{shockmodel} can be substitutes into the dynamic 
panel model \eqref{DPM} which yields,
\begin{align*} 
  y_{i,t} &= \eta_i + \alpha_iD_{i,t} + \phi_i y_{i,t-1} + \theta_i \x_{i,t} 
    + \varepsilon_{i,t} \\
  &= \eta_i + (\mu_{\alpha_i} + \beta\x_i + \gamma_{i})D_{i,t} 
    + \phi_i y_{i,t-1} + \theta_i \x_{i,t} + \varepsilon_{i,t} \\
  &= (\eta_i + \mu_{\alpha_i}D_{i,t}) + \phi_i y_{i,t-1} 
    + (\theta_i + \beta D_{i,t})\x_{i,t} + (\gamma_{i}D_{i,t} 
    + \varepsilon_{i,t}) \\
  &= \tilde\eta_i + \phi_i y_{i,t-1} + \tilde\beta_{i,t}\x_{i,t} + \tilde\varepsilon_{i,t},
\end{align*}
where $\tilde\eta_i \sim N(\mu_{\alpha_i},\sigma^2_\eta)$ 
and $\tilde\varepsilon_{i,t} \sim N(0, \sigma^2 + D_{i,t}\sigma^2_\gamma)$.

\end{document}

