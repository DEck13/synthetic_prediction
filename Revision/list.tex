\documentclass[12pt]{article}
\usepackage{amsmath}
\usepackage{amsthm}
\usepackage{amsfonts}
\usepackage{amscd}
\usepackage{amssymb}
\usepackage{natbib}
\usepackage{graphicx,times}
\usepackage{mathtools}
\usepackage{url}
%\usepackage{setspacing}
\usepackage{fullpage}
\usepackage{palatino}
\usepackage{mathpazo}
\usepackage{enumitem}
\usepackage{tcolorbox}
\usepackage{latexsym}
\usepackage{graphicx}
\usepackage{xifthen} % Allows us to put in optional arguments to questions
\usepackage{color}
\usepackage{manfnt}
\usepackage{ifthen}
\usepackage{listings}
\usepackage{nicefrac,mathtools}
\usepackage[top=2.2cm,bottom=2.2cm,right=2.1cm,left=2.1cm]{geometry}
\usepackage{color}   %May be necessary if you want to color links
\usepackage{hyperref}
\usepackage{float}
\usepackage{showexpl}
\usepackage{xcolor}
\usepackage{mathtools}
\usepackage{caption}
\usepackage{chngcntr}
\usepackage{xparse}
\usepackage{etoolbox}
\usepackage{blkarray}
\usepackage{cprotect}

\DeclarePairedDelimiter\abs{\lvert}{\rvert}
\DeclarePairedDelimiter\norm{\lVert}{\rVert}


\newcommand{\R}{\mathbb{R}}
\newcommand{\ds}{\displaystyle}
\newcommand{\mustar}{\mu^{\textstyle{*}}}
\newcommand{\betastar}{\hat{\beta}^{\textstyle{*}}}
\newcommand{\betahat}{\hat{\beta}}
\newcommand{\betaw}{\hat{\beta}_w}
\newcommand{\betastarT}{\hat{\beta}^{\textstyle{*}^T}}
\newcommand{\betabar}{\bar{\beta}^{\textstyle{*}}}
\newcommand{\bstar}{b^{\textstyle{*}}}
\newcommand{\vstar}{v^{\textstyle{*}}}
\newcommand{\Bstar}{B^{\textstyle{*}}}
\newcommand{\wstar}{w^{\textstyle{*}}}
\newcommand{\BICstar}{\textsc{bic}^{\textstyle{*}}}
\newcommand{\minBICstar}{\min_{s=1,...,r}\left\{\BIC^{\textstyle{*}}(s)\right\}}
\newcommand{\sstar}{s_m^{\textstyle{*}}}
\newcommand{\epstar}{\varepsilon^{\textstyle{*}}}
\newcommand{\epstarT}{\varepsilon^{{\textstyle{*}^T}}}
\newcommand{\epresstar}{\widehat{\varepsilon}^{\textstyle{*}}}
\newcommand{\epresstarT}{\widehat{\varepsilon}^{\textstyle{*}^T}}
\newcommand{\residstar}{\widehat{\varepsilon}^{\textstyle{*}}}
\newcommand{\Ystar}{Y^{\textstyle{*}}}
\newcommand{\Xstar}{X^{\textstyle{*}}}
\newcommand{\Wstar}{W^{\textstyle{*}}}
\newcommand{\Wstarinv}{W^{{\textstyle{*}^{-1}}}}
\newcommand{\Zstar}{Z^{\textstyle{*}}}
\newcommand{\YstarT}{Y^{{\textstyle{*}^T}}}
\newcommand{\XstarT}{X^{\textstyle{*}^T}}
\newcommand{\Sigstar}{\widehat{\Sigma}^{\textstyle{*}}}
\newcommand{\Sigstarhalf}{\widehat{\Sigma}^{\textstyle{*}^{1/2}}}
\newcommand{\Sigstarhalfinv}{\widehat{\Sigma}^{\textstyle{*}^{-1/2}}}
\newcommand{\lstar}{l^{\textstyle{*}}}

\newcommand{\Astar}{A^{\textstyle{*}}}
\newcommand{\Gostar}{\widehat{G}_o^{\textstyle{*}}}
\newcommand{\GostarT}{\widehat{G}_o^{\textstyle{*}^T}}
\newcommand{\Gohat}{\widehat{G}_o}
\newcommand{\GohatT}{\widehat{G}_o^T}
\newcommand{\Sigresstar}{\widehat{\Sigma}^{\textstyle{*}}}

\newcommand{\B}{\mathcal{B}}
\newcommand{\Lnorm}{\mathcal{L}}
\newcommand{\Sub}{\mathcal{S}}
\newcommand{\Q}{\mathcal{Q}}
\newcommand{\Proj}{\mathcal{P}}
\newcommand{\Env}{\mathcal{E}}
\newcommand{\Envspace}{\Env_{\Sigma}(\B)}
\newcommand{\utrue}{u_{\text{true}}}
\newcommand{\BIC}{\textsc{bic}}
\newcommand{\dimEnv}{\text{dim}\{\Envspace\}}
\newcommand{\minBIC}{\min_{s=1,...,r}\left(\BIC(s)\right)}
\newcommand{\nboot}{n_{\text{boot}}}

\newcommand{\X}{\mathbb{X}}
\newcommand{\Y}{\mathbb{Y}}
\newcommand{\Ymatstar}{\Y^{\textstyle{*}}}
\newcommand{\YmatstarT}{\Y^{\textstyle{*}^T}}
\newcommand{\Xmatstar}{\X^{\textstyle{*}}}
\newcommand{\XmatstarT}{\X^{\textstyle{*}^T}}

\newcommand{\Sigres}{\widehat{\Sigma}}
\newcommand{\SigY}{\widehat{\Sigma}_{Y}}
\newcommand{\SigX}{\widehat{\Sigma}_{X}}
\newcommand{\Pu}{\widehat{\Proj}_{\Env_u}}
\newcommand{\Pj}{\widehat{\Proj}_{\Env_j}}
\newcommand{\Qu}{\widehat{\Q}_{\Env_u}}
\newcommand{\Qj}{\widehat{\Q}_{\Env_j}}
\newcommand{\PD}{\widehat{\Proj}_{D}}
\newcommand{\betau}{\hat{\beta}_u}
\newcommand{\betaj}{\hat{\beta}_j}
\newcommand{\res}{\widehat{\varepsilon}}

\newcommand{\sestar}{\text{se}^{\textstyle{*}}\{\text{vec}(\hat{\beta})\}}
\newcommand{\setrue}{\text{se}_{\text{true}}\{\text{vec}(\hat{\beta})\}}

\newcommand{\vecop}[1]{\text{vec}\left( #1 \right)}
\newcommand{\vechop}[1]{\text{vech}\left( #1 \right)}
\newcommand{\indep}{\rotatebox[origin=c]{90}{$\models$}}

\DeclareMathOperator{\tr}{tr}
\DeclareMathOperator{\Var}{var}

\newtheorem{lem}{Lemma} 
\newtheorem{thm}{Theorem} 
\allowdisplaybreaks

\setlength{\parindent}{0cm}


\newcommand{\response}[1]{\noindent \textcolor{blue}{\emph{Response:} #1}}
\cMakeRobust\response

\hypersetup{
  linkcolor  = blue,
  citecolor  = blue,
  urlcolor   = blue,
  colorlinks = true,
} % color setup


\title{Referee report checklist}
\author{Jilei Lin and Daniel J. Eck}
\date{}

\bibliographystyle{plainnat}

\begin{document}

\maketitle

Thank you for the opportunity to resubmit at \emph{International Journal of Forecasting}. Thank you to the reviewers for comments that led to an improved version of this manuscript. We address all of the reviewers' comments below. An outlines of each change made or a rebuttal is provided for all of the reviewer's comments.

\section*{Editors}

---------------------------------------------------------------------------------------------------------------------------


We have now received feedback on your manuscript from two reviewers as well as an Associate Editor. The reviewers' reports are attached to this message, while the AE's comments are included below this letter.

Both reviewers and the AE agree that you address a relevant and interesting topic, and that your paper has the potential to provide a valuable addition to the forecaster's toolkit. That being said, the current version of the manuscript does not meet the publication standards of the IJF. A number of important points are raised by the reviewers that are either unclear or unresolved, and these need to be addressed. In addition, the Associate Editor provides some useful comments and suggestions, also pointing out that quite a substantial revision is required.

In sum, I have decided to give you the opportunity to revise and resubmit your manuscript. Please do note that this is a "major revision". This means that, while we feel that the work presented in the manuscript might be of interest for the journal, re-submitting your manuscript does not necessarily mean it will be accepted, eventually.

---------------------------------------------------------------------------------------------------------------------------

\response{Thank you for the opportunity to submit a revision. We include a point by point response for the comments and concerns raised by the reviwers.}

\section*{Associate Editor}

{\bf Comment:} Firstly, this contribution seems to me to be related to two exisiting strands of literature, neither of which is cited. Describing the contribution relative to these literatures might help some readers better appreciate the achievements.
These are the literatures on "ad hoc" model adjustments for breaks, see e.g.,  \cite{clements1996intercept} for an early contribution, and \cite{castle2015robust} for a more recent example. The second is to do with outliers (innovation, additive and level shifts) in ARMA models, see e.g., \cite{tsay1986time}. Comparing/contrasting the approach in the paper with these literatures may be illuminating. \\

\response{Thanks for pointing it out. The discussion of those papers is included in the introduction section of the revised manuscript.}\\

{\bf Comment 2:} Secondly, a single empirical example is presented. I would like to see the results for a range of series - this would generate confidence in the proposed approach. \\

\response{We include another example of forecasting the stock price of Apple Inc. during MacBook release in 2020 in Section 5.2 of the revised manuscript. Our method nearly perfectly recovers its stock price of the day after the release date. The donor pool constructed includes three time series of  Apple inc.'s stock price during MacBook release in the past three years. In this example, the shock is with precedent and our method works well because Apple Inc. is a well-established company that is not expected to reform its products a lot. As a result, knowledge from past shocks can bring insights into the future shocks. This example is representative for applying our methods to the time series that are to experience a shock that is with precedent in the past. In contrast, the example of Conoco Phillips is without precedent. In the past, Conoco Phillips  did not experience oil supply shock and financial crisis shock caused by  pandemic. In this example, our method is better than the original forecast though it cannot fully recover the stock price. We hope that the addition of the Apple example helps to generate confidence in our approach.
}

\newpage
\section*{Reviewer 1: \texttt{Cover letter for authors.pdf}}

Thank you very much for taking the time to provide a thoughtful review of our manuscript. We think that these comments will go a long way to improve the exposition of our manuscript. \\


%% addressed
{\bf  Comment 1:} How could the method be applied to forecast two or more period after a shock? \\

\response{Thanks for pointing it out. The method can be extended to forecast two or more periods after a shock. There are two situations.  \\

First, suppose that there is no additional shock after $T_1^* + 1$. From equation (1) in the manuscript, the forecast at time point $T_1^*+2$ is
\begin{align*}
 \hat{y}_{i, T_i^* + 2} = \hat{\eta}_1 + \hat{\phi}_i \hat{y}_{1, T_1^*+1}^{2} + \hat{\theta}_i'\mathbf{x}_{1, T_1^*+2}, \tag{*}
\end{align*}
where $\hat{y}_{1, T_1^*+1}^{2}=\hat{y}_{1, T_1^*+1}^{1} + \hat{\alpha}_{T_1^* + 1}$, where $\hat{y}_{1, T_1^*+1}^{1}$ is the forecast without adjustment, and  $\hat{\alpha}_{T_1^* + 1}$ is the shock-effect estimator for the shock at $T_1^*+1$. The forecast for $d>2$ more periods is
\begin{align*}
 \hat{y}_{i, T_i^* + d} = \hat{\eta}_1 + \hat{\phi}_i \hat{y}_{1, T_1^*+d-1}+ \hat{\theta}_i'\mathbf{x}_{1, T_1^*+d},
\end{align*}
where $\hat{y}_{1, T_1^*+d-1}$ can be also computed recursively using the above formula and we  emphasize  $ \hat{y}_{i, T_i^* + 1}=\hat{y}_{1, T_1^*+1}^{2}$. Since we use AR(1) model, $|\phi_1 |< 1$ is assumed.  As one can expect, the impact of the shock  $\alpha_1$ will diminish as $d$ increases. Similar logic can apply for AR($p$). \\

Second, suppose that  there is one more shock effect after $T_1^*+1$, say, $\alpha_{T_1^* + d^*}$ for some $d^* > 1$. Without loss of generality, assume $d^* = 2$. Under our framework, users may want to construct another set of donor pool with time series that experienced shock effects similar to $\alpha_{T_1^* + 2}$. Next,  apply our proposed method to compute $\hat{\alpha}_{T_1^* + 2}$, the estimator for the shock effect at $T_1^* +2$. The forecast at the point $T_1^* + 2$ will be,  
\begin{align*}
 \hat{y}_{i, T_i^* + 2} = \hat{\eta}_1 + \hat{\phi}_i \hat{y}_{1, T_1^*+1}^{2} + \hat{\theta}_1'\mathbf{x}_{1, T_1^*+2} + \hat{\alpha}_{T_1^* + 2}.
\end{align*}
These two cases can be further combined to generalize forecasting for two or more periods. It is added to Section 6.\\
}

{\bf Comment 2:} Page 3, line 30: \citet{blundell1998initial} model is a fixed effect dynamic panel and estimate the parameters using the panel analysis. It would be nice if you clarify in what sense your work is similar to them. \\

\response{The model of \citet{blundell1998initial} is
\begin{align*}
  y_{it} = \alpha y_{i, t-1} + \beta_1'x_{it} + \beta_2'x_{it-1} + \eta_i + v_{it}, \quad i = 1, \ldots, N, \quad t = 2, \ldots, T.
\end{align*}
where $|\alpha|<1$. In contrast, our model is
\begin{align*}
 y_{i,t} =\eta_i +\alpha_i D_{i,t} + \phi_i y_{i, t-1} + \theta_i'\mathbf{x}_{i,t} + \varepsilon_{i,t},
\end{align*}
where $|\phi_i| < 1$. Our model is similar to the one of \citet{blundell1998initial}  in the sense that we have the fixed-effect decomposition of the error term $\eta_i +\varepsilon_{i,t}$ and the autoregressive structure. However, our model allows for $\eta_i$ as random effects, different time lengths for each time series, different autoregressive parameters, and a shock component. To avoid confusion, we  remove this reference in the revised manuscript.}\\

{\bf Comment 3:} Page 4, line 50: Your paper considers forecasting a time-series model. It is confusing for readers to read your sentence ”In this article, we consider a dynamic panel data model...''.\\

\response{Sorry for the confusion. It is corrected  in the revised version.}\\

{\bf  Comment 4:} Figures on page 5: Figure labels, (a) and (b), are missing. What does the black line in the top graph represent? Is it the realized observations?\\

\response{Thanks for pointing it out. The labels (a) and (b) are now added in the revised manuscript.   The \textcolor{black}{black} line stands for the realized observations whereas the \textcolor{magenta}{magenta} line stands for the fitted values.} \\

{\bf Comment 5:} Page 7, line 52: Are these time series estimates or panel data estimates? If panel estimators are used, then OLS estimators in this model results in inconsistency of the estimators. In that case, how wouldn't this inconsistency affect the results in section 3 about the unbiasedness? \\

\response{Sorry for not making it clear. They are time series estimates or the ordinary least squares (OLS) estimates but not panel data estimates. In fact, our method can be generalized to any model whose parameter can be estimated in an unbiased fashion. In the second paragraph of Section 6, we wrote \\


\begin{tcolorbox}
``Although our work is developed for time-series or AR($p$) models, in fact, it can be generalized to any similar setting with a model of the response, whose parameters can be estimated \emph{\textcolor{red}{unbiasedly}}, an additive shock-effect structure, and the structure that the time series in the donor pool are independent of the one of interest.''
\end{tcolorbox}


\vspace{0.4cm}In other words, if the estimator used in panel data is unbiased, our method can adapt to this scenario by weighting the shock effects estimated by panel data estimation procedure. In this case, the propositions in Section 3 still hold though the exact variance expressions will change since we are not using OLS estimators.}\\


{\bf Comment 6:} Page 8, line 46: It is not clear to met how you estimate $\alpha_i$'s by OLS. Could you give the expressions for $\hat{\alpha}_i$? Also, how can you identify $\alpha_i$ from $\eta_i$? Isn't $\hat{\alpha}_{\rm adj}$ an estimator of $E(\alpha_1)$? \\

\response{Sorry for not making it clear. The estimation procedures can be explained in detail as follows. Note that for $i = 2, \ldots, n+1$, the design matrix for $i$th time series is $(\mathbf{1}_{T_i}, \mathbf{D}_{i}, \mathbf{X}_{i})$, where $\mathbf{X}_i$ is the covariates  of $i$th time series and $\mathbf{D}_{i}=(D_{i,t})_{t=1}^{T_i}$ with $D_{i,t}=I(t=T_i^*+1)$. We define $\mathbf{M}_i = (\mathbf{1}_{T_i}, \mathbf{D}_{i}, \mathbf{X}_{i})$. Then the expression for $\hat{\alpha}_i$ is
\begin{align*}
  \hat{\alpha}_i = [(\mathbf{M}'_i\mathbf{M}_i)^{-1}\mathbf{M}_i'\mathbf{Y}_{i}]_{2}, \quad 
  \text{ where }\quad \mathbf{Y}_{i}=(y_{i,t})_{t=1}^{T_i} 
 \quad  \text{ for } i = 2, \ldots, n+1.
\end{align*}
Note that $[\cdot]_{2}$ denotes the second element of a vector. That is, $\hat{\alpha}_i$ is the second element of the OLS estimate vector. Our model is
\begin{align*}
  y_{i,t} = \eta_i + \alpha_i D_{i,t} + \phi_i y_{i,t-1} + \theta_i'\mathbf{x}_{i,t} + \varepsilon_{i,t}, 
  \quad i = 1, \ldots, n+1.
\end{align*}
$\eta_i$ represents the effect of the intercept. Thus, we cannot identify $\alpha_i$ from $\eta_i$. As defined in (4) in the manuscript, through averaging, $\hat{\alpha}_{\rm adj}$ is an estimator  of $E(\alpha_1)$.} \\


{\bf Comment 7:} Page 9, line 10: It is not clear to me what $\mathbf{U}_i$'s are. An example could help understanding that better. \\


\response{Thanks for pointing it out. We  clarify it in the revised version, and change the notation to be $\mathbf{M}_i$, which represents the design matrix of the model used in OLS estimation at the $i$th time series. For example, suppose the sample size for $i$th time series in the donor pool (i.e., $i\geq 2$) is 4; we have one covariate consisting of $x_1, \ldots, x_4$; and the shock occurs at $t = 4$. In this case, $\mathbf{U}_i$ is 
\begin{align*}
  \mathbf{M}_i = \begin{pmatrix}
    1 & 0 & x_1 \\
    1 & 0 & x_2 \\
    1 & 0 & x_3 \\
    1 & 1 & x_4
  \end{pmatrix}.
\end{align*}
Suppose the sample size for the time series of interest is also 4, and we have a covariate $k_1, \ldots, k_4$. Note that the design matrix of the time series of interest will not contain $D_{1,t}$ since $D_{1,t}= I(t=T_1^*+1)=0$ due to we do not observe the shock. In this case, 
\begin{align*}
  \mathbf{M}_1 = \begin{pmatrix}
    1 &  k_1 \\
    1 &  k_2 \\
    1 &  k_3 \\
    1 &  k_4
  \end{pmatrix}.
\end{align*}} \\

{\bf Comment 8:} Page 9, line 15: Why the closed form expressions for $E(\hat{\alpha}_{\rm IVW})$ and $Var(\hat{\alpha}_{\rm IVW})$ are not provided? \\

\response{In page 8, we wrote \\
\begin{tcolorbox}
''The inverse-variance weighted estimator is defined as 
\begin{align*}
  \hat{\alpha}_{\rm IVW} = \frac{\sum_{i=2}^{n+1} \hat{\alpha}_i / \hat{\sigma}_{i\alpha}^2}{\sum_{i=2}^{n+1} 1/\hat{\sigma}_{i\alpha}^2},
  \quad \text{ where } \quad  \hat{\sigma}_{i\alpha}^2 = \hat{\sigma}^2_i( \mathbf{M}_i'\mathbf{M}_i)_{22}^{-1},
\end{align*}
where  $\hat{\alpha}_i$ is the OLS estimator of $\alpha_i$, 
$\hat{\sigma}_i$ is the residual standard error from OLS estimation, 
and $\mathbf{M}_i$ is the design matrix for OLS with respect to time series 
for $i = 2, \ldots, n+1$. Note that since $\sigma$ is unknown, estimation 
is required and the numerator and denominator terms are dependent in general.''
\end{tcolorbox}
To be more specific, $\hat{\sigma}_{i\alpha}$ depends on $\hat{\alpha}_i$. As a result, the numerator and denominator of $ \hat{\alpha}_{\rm IVW}$ are dependent. Without making distributional assumptions, it is impossible to evaluate $E(\hat{\alpha}_{\rm IVW})$ and $Var(\hat{\alpha}_{\rm IVW})$ in a closed form.}\\

{\bf Comment 9:} Page 13, line 14: $\mu_{\alpha}$ is unknown, and in practice we have to estimate it. Would the results hold if one replaces $\mu_{\alpha}$ with its estimate? \\

\response{We go for a plug-in approach. If we can estimate $\mu_{\alpha}$ well, the results should hold with high probability. Section 3.2 is dedicated to this estimation problem. Using plug-in approaches, we get a decision rule to determine whether the scalar adjustment will be beneficial with correctness probability $p$. Section 3.3 details the leave-one-out cross validation procedures, which are tailored to our setup, to estimate $p$. As a result, users can be informed about the credibility of our methods} \\


{\bf Comment 10:} Page 14, line 7: Shouldn’t $E(\alpha_1)$ be $\mu_{\alpha}$? Previously $\mu_{\alpha}$ was used.\\

\response{In the line 7 at Page 14, $E(\alpha_1)$ is taken under $\mathcal{M}_{21}$ or $\mathcal{M}_{22}$. In contrast, $\mu_{\alpha}$ is used under $\mathcal{M}_{1}$. In Section 2.1, we define the distribution of $\alpha_i$ as
\begin{align*}
  &\alpha_i \sim \mathcal{F}_{\alpha} 
 \quad  \text{ with }\quad  E_{\mathcal{F}_{\alpha}}(\alpha_i) =\mu_{\alpha}, Var_{\mathcal{F}_{\alpha}}(\alpha_i)=\sigma^2_{\alpha},\quad   \text{iid under } \mathcal{M}_1\\
 & \alpha_i = \mu_{\alpha} +\delta_i'\mathbf{x}_{i, T_i^*+1} + \tilde{\varepsilon}_i\quad  \text{iid under } \mathcal{M}_{21} \text{ or } \mathcal{M}_{22}.
\end{align*}
In other words,  $\mathcal{M}_{21}$  or $\mathcal{M}_{22}$ is an extended version of $\mathcal{M}_{1}$ by allowing it to depend on some covariates. However, in general, $E(\alpha_1)\neq \mu_{\alpha}$ under   $\mathcal{M}_{21}$  or $\mathcal{M}_{22}$ due to the presence of $\delta_i'\mathbf{x}_{i, T_i^*+1}$.} \\

{\bf Comment 11:} Page 18, numerical example: The model setup does not contain the individual effects (intercepts) similar to the models in equations (1)-(2) on page 6. How would the result change if you include them with different variances? \\

\response{The result wouldn't change. In fact, in our proof listed in the Appendix, we do not make use of the assumption that $\eta_i$ follows the same distribution. In other words, the results  still hold when the individual effects have different variances.}\\


{\bf Comment 12:} Page 23, construction of donor pool: Your model assumes that the shocks have the same distribution. Given that these shocks are from different sources and happened because of various reasons, how could one justify them? Practically, there are many previous shocks. How can someone choose from them?\\

\response{It is a good question. To clarify, $\mathcal{M}_1$ assumes that the shocks have the same distribution. However, $\mathcal{M}_{21}$ assumes the shocks have the same variance but have different conditional means with allowing covariates to play a role. Moreover, $\mathcal{M}_{22}$ assumes they have different conditional means and conditional variances by allowing $\delta_i$ to be random. But the shock effects are assumed to follow a general family of  unknown distribution with possible different means and variances.\\

$\mathcal{M}_{21}$ and $\mathcal{M}_{22}$ are constructed for generalization. In practice, it is hard to verify those assumptions. It leaves the discretion for users to construct donor pool by gathering time series experiencing similar events to the one the time series of interest experienced. \\

Theoretically, it is possible to verify the assumptions of $\mathcal{M}_{1}$ using empirical Bayes methods to some degree. First, we need a  pool of shock effects that users believe they are realizations from a common distribution, say, $\mathcal{F}_{\alpha}$. Using empirical Bayes methods, we can estimate $\mathcal{F}_{\alpha}$ to have $\hat{\mathcal{F}}_{\alpha}$. To verify whether the remaining shock effects are from $\mathcal{F}_{\alpha}$, we can check the quantile of $\hat{\mathcal{F}}_{\alpha}$ to judge whether they are extreme in $\hat{\mathcal{F}}_{\alpha}$. However, the theoretical justification for this approach is left for future research. \\

In practice, someone can select the shocks that are similar to the one of interest. For example, they can be similar in causes, scale, and characteristics. Using our data analysis example, on March 9th 2020, Conoco Phillips experienced a shock during COVID-19 pandemic and oil supply shock. During COVID-19 pandemic, the US economy is in a situation that resembles the one in financial crisis. As a result, we select the shocks in the past that are related to financial crisis and oil supply shock.} \\


{\bf Comment 13:} Figure 2 on Page 24: Are the black dots the real data? What does the line represent?\\

\response{Thanks for pointing it out. We  clarify this in the revised version. The black dots represent the real data. The line in \textcolor{magenta}{magenta} color represents the fitted line of our model.}

\newpage

\section*{Reviewer 2:  \texttt{MinimizingPostShockForestError\_IJF\_pdf}}

Thank you for your appreciation of the potential of our methods, and for your valuable detailed comments.  We think these comments will help a lot in improving our manuscript.\\

{\bf Comment 1:} The method hinges on the assumption that the time series forming the donor pool are independent from the time series under study prior to the shock. This seems to be a very restrictive assumption. The shocks that the donor pool have undergone in the past must be representative of the shock that the time series of interest is experiencing, but the series are independent. Given this restrictive assumption what practical applications are ruled out? \\

\response{Yes. It could be a restrictive assumption in a number of situations. For example, if time series in a donor pool are observed in almost concurrent time as the one under study (or, the time differences are not large enough), the independence assumption may not hold. As a result, application in panel data should be ruled out. However, if the time differences are large, the dependence should diminish as time passes.} \\

{\bf Comment 2:} It would be helpful to the reader to be very explicit about the timing of when information is realized. Is it correct that the researcher knows at $T_1^*$ that there will be a shock at $T_1^* + 1$, and they know the type of shock that it will be in order to identify the relevant donor pool shocks, but they do not know the magnitude? This is quite an informational advantage over unanticipated shocks. In many situations one could imagine that the shock was unanticipated. Therefore, at $T_1^* + 1$  there is a large forecast error (your model 1) but now information is available about the type of shock, so the proposed methodology of the shock effect estimator is applied at $T_1^* + 2$, yielding gains over the unadjusted forecast. The timing of information is crucial, but the methodology proposed in the paper seems to be applicable in many different timing scenarios. \\

\response{It is correct that the researcher knows at $T_1^*$ that there will be a shock at $T_1^* + 1$. In practice, it usually happens to the case that when we realize that the shock is about to occur, we  know the \emph{general type}  or \emph{general cause} of the shock. The magnitude is not known. \\

For example, in the Conoco Phillips stock example, when we knew the outbreak of COVID-19 was about to come, past shocks related to financial crisis may be helpful. Or, near the outbreak of COVID-19, suppose that a department in a university realized that it would be faced with declining applications. It could make use of the information about the decline of its admitted students in the past or that of other departments when similar situations (i.e., the number of applications suddenly decreases) happened.\\

We admit it is quite an informational advantage over unanticipated shocks. However, in practice, if a shock is unanticipated, we may not even be able to construct such a post-shock analysis since we do not know it is about to happen.\\

You are right about the timing. It can be applicable in many different timing scenarios. Generally, suppose that we know at $T_1^*$ that a shock is about to occur at $T_1^* + d$ for $d\geq 1$ and $d\in \mathbb{N}$ and that we have information about the future covariates for $t = T_1^* +1 , \ldots, T_1^* + d$. It is possible to generalize our method to this situation. \\

Thank you for your remark on the timing scenarios. It is very helpful for improving our manuscript.} \\

{\bf Comment 3:} Related to the point above, the model applied is an AR(1) with contemporaneous regressors. This structure requires the covariates to be known in the future for \emph{ex ante} forecasts, or for them to be forecast. The set-up assumes θ1 to be well behaved and the covariates are not subject to shocks if they need to be forecast. Empirically, we could also think of cases where the mean shift $D_{i,t}$ interacted with the covariates which would give an additional forecast error via $\mathbf{x}_{1, T_1^* + 1}$. The theoretical results seem to hinge on known $\mathbf{x}_{1,t}$ entering contemporaneously to derive unbiasedness. What are the implications of the covariates needing to be forecast? \\

\response{It is a very good point. Thanks for pointing it out. Indeed, our method assumes $\mathbf{x}_{1, T_1^* + 1}$ to be given. If it is to be forecasted with errors, it will possibly result in a biased estimation of $E(\alpha_1)$ with $\hat{\alpha}_{\rm wadj}$ such that the evaluation of the proposition may not be reliable. \\

In fact, the situation when  covariates are forecasted with errors is similar to the one when the variance of the shock effect is large. In both situations, the signal from covariates is weak. Our simulation discusses this situation by increasing $\sigma_{\alpha}$. It shows that a slightly weak signal will not compromise the correctness of our evaluated propositions a lot. However, if the forecast for the covariates is seriously bad (i.e., the signal from covariates is really weak), our result shows that the risk-reduction evaluation may be only as good as a random guess.} \\

{\bf Comment 4:} Given the dynamic structure, what happens after $T_1^*+1$? Couldn't you use the donor pool shocks to estimate the shock transition (e.g. apply the shock effect estimator to the m periods following the shock), and produce an intercept correction dummy for the next $m$ periods? It would be interesting empirically to see if the shock is anticipated to be permanent or transitory. I wonder if the theoretical statements of when forecast risk is minimized may be adapted to assess when transitory shocks die out such that the unadjusted forecast is preferable again? \\

\response{In our setup, we only model the time series under study up to $t = 1, \ldots, T_1^*+1$.  We do not know what happened  after $T_1^*+1$. It is possible that there are no shocks or additional shocks after $T_1^*+1$. \\

It is possible to use the donor pool to estimate the shock transition,  and produce an intercept dummy for the next $m$ periods. Assume there are no additional shocks  after $T_1^*+1$, and $T_i^*+m \leq T_i$ for $i = 2, \ldots, n+1$. Consider the model
\begin{align*}
  y_{i,t} = \eta_i + \theta_i'\mathbf{x}_{i,t} + \phi_i y_{i,t-1 }+\beta_{i, T_i^* + d}I(t = T_i^*+d) +\varepsilon_{i,t}
\end{align*}
for $i = 2, \ldots, n+1$ and $d = 1, \ldots, m$. Suppose we have information of the covariates up to $T_1^*+d$ either by given information or by forecasting. We can use our proposed synthetic control method to compute weights $\mathbf{W}^*$ and OLS to estimate $\beta_{i, T_i^*+d}$ to compute $\hat{\beta}_{1, T_1^*+d}$.  As a result, we obtain $\hat{\boldsymbol{\beta}}_1^{(m)} = (\hat{\beta}_{1, T_1^*+d})_{d = 1}^m$. However, it cannot be a simple intercept dummy but instead an iterative adjustment in computing $m$-ahead forecasts. For example, $\hat{y}_{1, T_1^*+2}^{(1)}=\hat{y}_{1, T_1^*+2}^{(0)}+ \hat{\beta}_{1, T_1^*+2}$, where $\hat{y}_{1, T_1^*+2}^{(0)}$ is the standard 2-ahead forecasts considering $\hat{y}_{1, T_1^*+1}^{(1)}$ in computation.  The approach of \cite{agarwal2020two} is an alternative  to this problem though its adjustment is without theoretical guarantees. \\

To judge if the shock is permanent, we may go for another approach. Consider the model
\begin{align*}
  y_{i,t} = \eta_i + \theta_i'\mathbf{x}_{i,t} + \phi_i y_{i,t-1 }+\beta_{i}I(T_i^*< t \leq  T_i^*+d) +\varepsilon_{i,t}.
\end{align*}
We can compute $\hat{\beta}_i$ from OLS and $\mathbf{W}^*$ from synthetic approach with covariates $d \times p$ (see for example of $p = 2$ in our added Apple example in Section 5.2 of revised transcript) to compute $\hat{\beta}_1$. Note that $\hat{\beta}_1$ is unbiased. We can use parametric double bootstrap to test the hypothesis that $H_0 \colon \beta_1=0$ versus $H_1 \colon \beta_1\neq 0$. \\

To our best knowledge, we don't know whether the theoretical statements of our proposed methods in the manuscript can be adapted to assess when transitory shocks die out.} \\

{\bf Comment 5:} While the simulations were useful, I found the discussion in section 4 quite impenetrable. There are so many decisions that a researcher would need to make, including which weighting method and all the bootstrapping and leave-one-out decisions. It would be very helpful to the reader if you could give a summary of all your results stating your preferred user decisions for generic specifications. \\

\response{There are mainly three decisions to make for an user: to choose 
  \begin{itemize}
    \item among  $\hat{\alpha}_{\rm adj}$, $\hat{\alpha}_{\rm IVW}$, and $\hat{\alpha}_{\rm wadj}$ 
    \item between $\mathcal{B}_f$ and $\mathcal{B}_u$ (see detailed definition in Section 3.2)
    \item The $k$ in the $k$ random draws in LOOCV
  \end{itemize}
  First, $\hat{\alpha}_{\rm adj}$ is usually recommended when we believe the shock is i.i.d from a distribution. Second, $\hat{\alpha}_{\rm IVW}$ is recommended when the time length $T_i$ in the donor pool varies vastly across the donor pool. Third, $\hat{\alpha}_{\rm wadj}$  is recommended when we assume the shocks are from the same family of distribution, and the mean of the shock depends on some covariates. These recommendations are included in Remark 3 at the end of Section 2. \\
  
  The selection between $\mathcal{B}_f$ and $\mathcal{B}_u$ is more of a philosophical question. In Section 3.2,  we noted
  
  \begin{tcolorbox}
  $\mathcal{B}_u$ treats the donor pool as realizations from some infinite super-population of potential donors. In contrast, $\mathcal{B}_f$ treats the donor pool as being fixed  and known before the analysis is conducted, where the randomness arises from parameters and idiosyncratic error.   
  \end{tcolorbox}

In Section 4.3, we found that $\mathcal{B}_u$ is better than $\mathcal{B}_f$ when  the donor pool size is moderately small ($n< 10$) and the signal from the covariates is large. Otherwise, they are nearly the same. \\


At the end of Section 4.3, we noted that when $p < n$, there are infinitely many solutions to $\mathbf{W}^*$ such that $\mathbf{W}^*$ will take values on the boundary of $\mathcal{W}$, in which case bootstrapping may fail to estimate the distribution of $\hat{\alpha}_{\rm wadj}$ \citep{andrews2000inconsistency}. $\mathcal{B}_f$  is immune to this issue. However, our simulation results in the Supplementary Materials show that the non-uniqueness does not seriously compromise the inference. \\



In summary, $\mathcal{B}_u$ is preferred when  the donor pool size is moderately small ($n< 10$) and the signal from the covariates is \emph{believed} to be large. If $p < n$ and non-uniqueness problem is of concern due to theoretical reasons,  $\mathcal{B}_f$ is preferred. In other cases, the choice between  $\mathcal{B}_u$  and  $\mathcal{B}_f$ would be up to personal preferences related to philosophical differences mentioned above. This explanation is added to the end of Section 4.  \\

%Lastly,  greater number of bootstrap replication $B$ is preferred. Usually, $B = 200$ is sufficient. 
Ideally, $k=n$ is preferred. However, if the data size is large such that $k=n$ is computationally expensive, we found that $k = 5$ or $k = 10$ works well. In practice, $(k+1) B$ number of bootstrap replications are implemented to complete the evaluation, where $k B$ is from LOOCV to estimate the correctness parameter $p$ and $B$ is from ordinary bootstrap using the whole donor pool. \\

} \\

{\bf Comment 6:} Section 2 is hard to read without defining the terms that come in the second para- graph of section 2.1 (lines 19-23 of p.6). I'd suggest moving this to the beginning of section 2. \\

\response{Sorry for not making it clear. The terms not defined in the caption of Figure 1 are now defined in the revised manuscript. For other parts, we find there are no terms that are not defined. The beginning parts of Section 2 are intended to give a general picture to the reader that our method is in fact general whereas the setup we investigate in Section 2.1 is a special case. }

\bibliography{synthetic-prediction-notes}

\end{document}





