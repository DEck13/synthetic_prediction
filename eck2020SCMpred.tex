\documentclass[11pt]{article}
\usepackage[utf8]{inputenc}

\usepackage{amsmath}
\usepackage{amsthm}
\usepackage{amsfonts}
%\usepackage{amscd}
\usepackage{amssymb}
\usepackage{graphicx}
\usepackage{mathtools}
\usepackage{natbib}
\usepackage{url}

\usepackage{geometry}
\usepackage[usenames]{color}
\geometry{margin=1in}

\newcommand{\R}{\mathbb{R}}
\newcommand{\w}{\textbf{w}}
\newcommand{\x}{\textbf{x}}
\newcommand{\X}{\textbf{X}}
\newcommand{\Y}{\textbf{Y}}
\newcommand{\Hist}{\mathcal{H}}
\newcommand{\Tstar}{T^{\textstyle{*}}}
\newcommand{\Wstar}{W^{\textstyle{*}}}
\newcommand{\WstarI}{W^{I^{\textstyle{*}}}}
\newcommand{\WstarN}{W^{N^{\textstyle{*}}}}
\newcommand{\wstar}{w^{\textstyle{*}}}
\newcommand{\wstarI}{w^{I^{\textstyle{*}}}}
\newcommand{\wstarN}{w^{N^{\textstyle{*}}}}


\newcommand{\norm}[1]{\left\lVert#1\right\rVert}
\newcommand{\indep}{\perp\!\!\!\perp}

\DeclareMathOperator{\E}{E}
\DeclareMathOperator{\Var}{Var}

\newtheorem{cor}{Corollary}
\newtheorem{lem}{Lemma}
\newtheorem{thm}{Theorem}
\newtheorem{defn}{Definition}
\newtheorem{prop}{Proposition}

\newcommand\red[1]{{\color{red}#1}}

\allowdisplaybreaks

\title{Sythetic prediction of treatment effects before they are implemented}
\author{Daniel J. Eck and Soheil Eshghi}

\begin{document}


\maketitle

\begin{abstract}
The synthetic control method (SCM) is a powerful methodology for evaluating the effects of treatments over time. In this article we reformulate SCM so that it can estimate treatment effects before they are implemented. 
\end{abstract}


\section{Introduction}

\cite{athey2017state} state that SCM is ``arguably the most important innovation in the policy evaluation literature in the last 15 years.'' We reformulate the synthetic control method \citep{abadie2010synthetic} as a tool to estimate multiple counterfactual with the aim of estimating what a treatment effect would be. 





\section{Background on SCM}
What follows is adapted from \cite{abadie2010synthetic}: \\

Let $Y_{it}^N$ be the counterfactual outcome for subject $i$ at time point $t$ in the absence of treatment for subjects $i = 1,\ldots, n+1$, $t = 1,\ldots, T_i$.  Let $Y_{it}^I$ be the counterfactual outcome that would be observed for subject $i$ at time $t$ if subject $i$ is exposed to the treatment. We let $\Tstar_i$ be the time that the treatment is assigned to subject $i$ where $T_i = \infty$ for subjects that do not receive the treatment. It is assumed that $Y_{it}^N = Y_{it}^I$ for all $t < \Tstar_i$.

Let $\alpha_{it} = Y_{it}^I - Y_{it}^N$ be the treatment effect for individual $i$ at time period $t$, and let $D_{it}$ be an indicator that is 1 if individual $i$ is given the treatment and $t \geq \Tstar_i$. The observed outcome is 
$$
  Y_{it} = Y_{it}^N + \alpha_{it}D_{it}.
$$

In SCM the goal is to estimate $\alpha_{1\Tstar_1+1}, \ldots, \alpha_{1T_1}$ since subject $1$ is the only individual who receives the treatment. To achieve this goal we need to estimate $Y_{1t}^N$ for all $t > \Tstar_1$ since $Y_{1t}^I$ is observed over this time period. Suppose that $Y_{it}^N$ can be represented as a factor model,
$$
  Y_{it}^N = \delta_t + \theta_t'Z_i + \lambda_t'\mu_i + \varepsilon_{it}, 
$$
where $\delta_t$ is an unknown common factor with constant factor loadings across units, $Z_i \in \R^p$ is a vector of observed covariates (not affected by the treatment), $\theta_t \in \R^p$ is a vector of unknown parameters, $\lambda_t \R^F$ is a  vector of unobserved common factors, $\mu_i \in \R^F$ is a vector of unknown factor loadings, and the error terms $\varepsilon_{it}$ are unobserved errors with $E(\varepsilon_{it}) = 0$.

Let $W = (w_2,\ldots, w_{n+1}) \in \R^n$ be a vector of weights that satisfy $\sum_{i=2}^{n+1}w_i = 1$ and $w_i \geq 0$ for all $i = 2,\ldots, n+1$. The value of the outcome variable for each synthetic control indexed by $W$ is
$$
  \sum_{i=2}^{n+1}w_i Y_{it} = \delta_t + \theta_t'\sum_{i=2}^{n+1}w_iZ_i + \lambda_t'\sum_{i=2}^{n+1} w_i\mu_i + \sum_{i=2}^{n+1} w_i\varepsilon_{it},
$$
and this synthetic control is then used to construct an estimate of the intervention effect,
$$
  \widehat{\alpha}_{1t} = Y_{1t} - \sum_{i=2}^{n+1}w_i Y_{it}.
$$

The outcome variable of interest is observed for $T_i$ periods, $t = 1,\ldots,T_i$, for the subject affected by the intervention, $Y_{1t}$, and the unaffected subjects, $Y_{it}$, $i = 2,\ldots, n+1$. Let $K_i \in \R^{\Tstar_i}$ define a linear combination of preintervention outcomes: $\bar{Y}_i^K = \sum_{t=1}^{\Tstar_i}k_tY_{it}$. Consider $M$ such linear combinations $K_1, \ldots, K_M$, and let $\X_1 = (Z_1', \bar{Y}_1^{K_1}, \ldots, \bar{Y}_1^{K_M} )' \in \R^{k}$, $k = p + M$, be a vector of preintervention characteristics for the exposed subject. Similarly $\X_o \in \R^{k \times n}$ is a matrix containing the same variables for the unaffected subjects. That is, the $i$th column of $\X_o$ is $(Z_i',Y_i^{K_1},\ldots,Y_i^{K_M})'$ (assuming that $T_i = T$ for all $i$). The vector $\Wstar$ is chosen to minimize some distance, $\|\X_1 - \X_o W\|$. \cite{abadie2010synthetic} proposed using $\|\X_1 - \X_oW\|_V = \sqrt{(\X_1 - \X_oW)'V(\X_1 - \X_oW)}$, where $V \in \R^{k \times k}$ is some symmetric and positive semidefinite matrix. %If the relationship between the outcome variable and the explanatory variables in $\X_1$ and $\X_o$ is highly nonlinear and the support of the explanatory variables is large, interpolation biases may be severe.


\section{Synthetic prediction of treatment effect}

Here we ``reverse'' SCM and use it as a methodology for constructing both the intervention and the nonintervention for subject 1 at future time periods. In this context we suppose that we have $\Tstar_1$ preintervention observations of $Y_{1t}^N$ and no observations of $Y_{1t}$ for $t > \Tstar_1$. We aim to estimate $\alpha_{1t}$ for $t = \Tstar_1 + 1,\ldots, T_1$. We achieve this by using SCM to estimate $Y_{1t}^I$ and $Y_{1t}^N$ for$t = \Tstar_1 + 1,\ldots, T_1$.

We will now allow for two donor pools to estimate both counterfactuals $Y_{1t}^N$ and $Y_{1t}^I$ for $t > \Tstar_1$. We will let $n_I$ and $n_N = n - n_I$ be the number of subjects in the donor pool used to estimate $Y_{1t}^I$ and $Y_{1t}^N$ respectively. We estimate $Y_{1t}^I$ using data from the first donor pool. In this pool, we suppose that we have $\Tstar_i$ observations of $Y_{it}^N$ and $T_i - \Tstar_i$ postintervention observations of $Y_{it}^I$ for subjects $i = 2,\ldots,n_I+1$. We estimate $Y_{1t}^N$ using data from the second donor pool. In this pool, we suppose that we have $T_1 + 1$ observations of $Y_{it}^N$ %and $T_i - \Tstar_i$ postintervention observations of $Y_{it}^I$ 
for subjects $i = n_I+2,\ldots,n+1$. %where $T_i - \Tstar_i$ is allowed to equal 0.
Not all of the subjects are assumed to follow the same timelines as subject 1 in this context. We make the following assumption to ensure that the treatment assignments from subjects $2,\ldots,n+1$ do not interfere with the treatment effect on subject $1$.


\vspace*{0.5cm}
\noindent\textbf{Assumption 1 (Independence)} We assume that $(Y_{1t}^N,Y_{1t}^I,Z_1)$, $t = 1,\ldots,\Tstar_1$ are all independent of $(Y_{it}^N, Y_{it}^I, Z_i)$, $t = 1,\ldots,T_i$, $i = 2,\ldots,n+1$. 
\vspace*{0.5cm}

For now, we estimate the treatment effect $\alpha_{1\Tstar_1+1}$, the treatment effect for the first time period in which no counterfactuals $Y_{1t}^I$ or $Y_{1t}^N$ are observed. 


Let $\X_1 = (Z_1', \bar{Y}_1^{K_1}, \ldots, \bar{Y}_1^{K_M} )' \in \R^{k}$, $k = p + M$, be a vector of preintervention characteristics for subject 1. Let $\X_o^I \in \R^{k \times n_I}$ be a matrix containing the same variables for the $n_I$ subjects belonging to the first donor pool. That is, the $i$th column of $\X_o^I$ is $(Z_i',Y_i^{K_1},\ldots,Y_i^{K_M})'$ (assuming that $T_i = T$ for all $i = 2,\ldots,n_I+1$). The vector $\WstarI$ is chosen so that  
$$
  \WstarI = \text{argmin}_{W} \sqrt{(\X_1 - \X_o^IW)'V(\X_1 - \X_o^IW)},
$$
where $V \in \R^{k \times k}$ is some symmetric and positive semidefinite matrix. Similarly, let $\X_o^N \in \R^{k \times n_N}$ be a matrix containing the same variables for the $n_N$ subjects belonging to the second donor pool. That is, the $i$th column of $\X_o^N$ is $(Z_i',Y_i^{K_1},\ldots,Y_i^{K_M})'$ (assuming that $T_i = T$ for all $i = n_I+2,\ldots,n+1$). The vector $\WstarN$ is chosen so that  
$$
  \WstarN = \text{argmin}_{W} \sqrt{(\X_1 - \X_o^NW)'V(\X_1 - \X_o^NW)}.
$$
To ensure that both donor pools can reliably estimate the treatment effect $\alpha_{1,\Tstar_1+1}$ we assume that no treatment selection bias exists.


\vspace*{0.5cm}
\noindent\textbf{Assumption 2 (No selection bias)} We assume that $(Y_{it}^N,Y_{it}^I) \indep D_{it}$ for all $t = 1,\ldots,T_i$ and all subjects $i = 1,\ldots n+1$.
\vspace*{0.5cm}

Under assumptions 1 and 2 we propose estimating $\alpha_{1,\Tstar_1+1}$ with 
$$
  \widehat{\alpha}_{1,\Tstar_1+1} = \sum_{i=2}^{n_I + 1} \wstarI_i Y_{i\Tstar_i+1} - \sum_{i=n_I + 2}^{n + 1} \wstarN_i Y_{i\Tstar_i+1}.
$$





\section*{Notes}
{\bf Taken from Soheil's notes (begin).}
Assume we have covariate-observation pairs $(y_i, \x_i)$ for $n$ points, and 
that $\x_1$, the covariates of the intervened unit, lies within the convex hull 
of $(\x_i)_{i=1}^n$. One way to match covariates is to pick a vector $\w$ such 
that $\w'1=1$ and $\norm{\x_1 - \X\w}$, where $\X$ is the matrix of available 
covariates (arranged in column form), is minimized.

One can incorporate a notion of locality by penalizing the use of covariates 
far from the covariates of interest, through incorporating the following term 
in either a constraint or as part of the objective:
$\sum_{i=1}^n \w_i \norm{\x_1 - \x_i}$.

This leads to the following optimization problem:
\begin{align*}
\min_{\w} ~~~& \norm{\x_1 - \X\w}\\
\text{s.t.}~~~& \w'1=1,\\
& \sum_{i=1}^n \w_i\norm{\x_1 - \x_i} \leq \delta, \\
& \text{Var}(\Y\w)\leq \gamma,
\end{align*}
with $\delta$ and $\gamma$ being rightsizing parameters. 

We can simplify the above using the following assumptions:

\ldots
\red{Once I have time I will write these out so that e can extract the fixed values mostly to one side.}


\red{The rest is kind of bad as is, to be honest, and hard to justify. I still think we should weigh down the ones that are pretty far from the expected value, but it's not clear why the variance condition above will not be enough and we have to do a prediction model for all $(n-1)-$ tuples. Any ideas?}

The use of the transformed linear model (linking covariates to observations of 
outcomes) to create the counterfactual show our belief in the model 
specification. Under these conditions, the prediction of the model for the 
outcome at covariates $\x_i$ based on all other observed covariates $\x_{-i}$ 
(if it is within the convex hull) should closely match its observed outcome.  
Any discrepancy should make us less confident in the observation of the 
outcome (if we hold the model specification to be correct), so we would want 
$y_i$ to play less of a role in the creation of the counterfactual in that 
scenario.  Thus, we will put an upper limit on $\w_i$, the weight assigned to 
the covariate-observation pair by a function of the discrepancy between $y_1$ 
and the created synthetic prediction for it $\hat{y}_i$ from $\x_{-i}$, i.e., 





 
{\bf Taken from Soheil's notes (end).} \\

\bibliographystyle{plainnat}
\bibliography{synthetic-prediction-notes}


\end{document}